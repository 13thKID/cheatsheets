\documentclass[10pt]{article}

\usepackage[utf8]{inputenc}
\usepackage[
    top=0.5cm,
    left=0.6cm,
    right=0.6cm,
    bottom=0.5cm,
]{geometry}

\usepackage[T1]{fontenc}

\usepackage{amsmath}
\usepackage{amssymb}
\usepackage{bm}% bold math
\usepackage{dcolumn}% Align table columns on decimal point
\usepackage{etoolbox}
\usepackage{float}
\usepackage{float}% bold math
\usepackage{graphicx}% Include figure files
\usepackage{ifthen}
\usepackage{indentfirst}
\usepackage{lipsum}
\usepackage{nameref}
\usepackage{sectsty}
\usepackage{setspace} % lineheight
\usepackage{subcaption}
\usepackage{wrapfig}

\usepackage{multicol}
\setlength{\columnsep}{.5cm}

\usepackage[most]{tcolorbox}
\newtcolorbox{mybox}[1]{colback=gray-0.7!5!white,colframe=gray-0.7!75!black,fonttitle=\bfseries,title=#1}
\newtcolorbox{summarybox}[1]{colback=teal!5!white,colframe=teal!75!black,fonttitle=\bfseries,title=#1}

\usepackage{xcolor}
\definecolor{violet}{HTML}{751aff}
\definecolor{gray-0.9}{gray}{0.9}
\definecolor{gray-0.8}{gray}{0.8}
\definecolor{gray-0.7}{gray}{0.7}

% \usepackage[font=footnotesize,labelfont=bf]{caption}
\usepackage[font=small,labelfont=bf]{caption}
\usepackage[colorlinks,citecolor=violet,urlcolor=violet,bookmarks=false,linkcolor=violet,hypertexnames=true]{hyperref}

\pagecolor{gray-0.9}
\pagenumbering{gobble}

% ==============================================================================
% NEW COMMANDS
% ==============================================================================

% \renewcommand{\baselinestretch}{1.4}
\newcommand{\code}[1]{{\color{teal}\texttt{#1}}}
\newcommand{\cmd}[3][.3]{\code{#2} : #3 \\[#1em]}
\newcommand{\opts}[2][.5]{\hspace*{.5cm}\begin{minipage}{0.9\textwidth}
  #2
  \vspace*{-1em}
\end{minipage}\\[#1em]}

\newcommand{\opt}[2]{
  \code{#1} : #2 \\
}

\newcommand{\cluster}[2]{\begin{mybox}{#1}
  #2
  \vspace*{-1.3em}
\end{mybox}}

\newcommand{\summary}[2]{\begin{summarybox}{#1}
  #2
  \vspace*{-1.3em}
\end{summarybox}}

\newcommand{\headline}[1]{\textbf{\color{teal}#1}}

\newcommand{\feature}[2]{
  \headline{#1}
  \vspace*{-.5em}
  \begin{itemize}
    \setlength\itemsep{-.1em}
    \small
    #2
  \end{itemize}
  \vspace*{.3em}
}

% ==============================================================================
% RE-NEW COMMANDS
% ==============================================================================

% \renewcommand*{\thesubfigure}{Figure \arabic{figure} (\alph{subfigure}):}

\renewcommand\thesection{\arabic{section}}

% ==============================================================================
% BEGIN DOCUMENT
% ==============================================================================

\begin{document}

\begin{multicols*}{2}

  \cluster{Containers}{
    \cmd{docker ps}{listing containers}
    \opts{
      \code{-a} : list all, running and stopped \\
    }
    \cmd{docker run <image>}{run an image}
    \opts{
      \code{-i} : interactive mode \\
      \code{-t} : expose TTY \\
      \code{-p} : expose port \code{ext:int} eg. \code{3000:80} \\
      \code{-d} : detached mode \\
      \code{-v <abs-lpath>:<cpath>} : create bind mount \\
      \code{-v <abs-lpath>:<cpath>:ro} : create read-only bind mount \\
      \code{-v <name>:<cpath>} : create named volume \\
      \code{-v <cpath>} : create anonymous volume \\
      \code{-rm} : automatically remove when exits \\
      \code{---env (-e) KEY=VALUE} : overwriting default env vars \\
      \cmd{---env-file <env-file-path>}{overwriting default env vars with vars from .env file}
      \code{-name <string>} : give container a name \\
      \cmd{---network=host}{Use host networking. \\ Then \code{-p, --publish, -P}, and \code{--publish-all} are ignored, since container does not have its own IP address}
      \cmd{---add-host=host.docker.internal:host-gateway}{Mapp DNS host.docker.internal to host-gateway}
    }
    \code{docker rm <container>} : remove container \\[.3em]
    \code{docker stop <container>} : stop running container \\[.3em]
    \code{docker start <container>} : start stopped container \\[.3em]
    \code{docker attach <container>} : attach to the running container \\[.3em]
  }

  \cluster{Images}{
    \cmd{docker build <src>}{build docker image; \code{src}(.) - usually the \code{Dockerfile} is in the root}
    \opts{
      \code{-t <name>:<tag>} : tag an image \\
      \cmd{---build-arg KEY=VALUE}{Overwriting default values of the args}
    }
    \code{docker images} : list all images \\[.3em]
    \code{docker rmi <image-id>} : remove image \\[.3em]
    \code{docker image inspect <image-id>} : details of the image\\[.3em]
    \code{docker image prune} : remove image \\[.3em]
    \opts{
      \code{-a} : remove all images \\
    }
  }

  \begin{mybox}{DockerHub \& Sharing}
    \cmd{docker push <image-name>}{Push image to DockerHub}
    \cmd{docker pull <image-name>}{Pull image from DockerHub}
    \cmd{docker login}{Login into DockerHub}
    \cmd{docker logout}{Logout from DockerHub}
    \vspace*{-1.3em}
  \end{mybox}

  \cluster{Volumes}{
    \cmd{docker volume ls}{List volumes}
    \cmd{docker volume rm}{Remove volume}
  }

  \cluster{Other}{
    \cmd{docker logs <container>}{fetch the logs of the container}
    \opts{
      \code{-f} : follow \\
    }
    \cmd{docker cp <src> <dest>}{copy files to / from the container; \code{src/dest} (\code{<container-name>:<path> | <path>})}
  }

  \begin{mybox}{Dockerfile}
    \code{FROM <base-image>} : eg. \code{node}  \\[.3em]
    \code{WORKDIR /app} : Commands will be executed relatively to this dir \\[.3em]
    \code{COPY <src> <src>} : \code{src} (\code{.}) \code{dest} (\code{/app}) \\[.3em]
    \code{RUN npm install} : runs when image is created \\[.3em]
    \cmd{ARG NAME=DEFAULT\_VAL}{Defining build argument}
    \code{ENV PORT 80} : Creates env variable PORT with a default value 80 \\[.3em]
    \code{EXPOSE \$PORT} : exposes port under the PORT variable \\[.3em]
    \code{VOLUME ["<path>", ...]} : create anonymous volume \\[.3em]
    \code{CMD ["node", "server.js"]} : runs when container is started. Should always be the last \\[.3em]
    \vspace*{-1.3em}
  \end{mybox}

  \cluster{Networking}{
    \cmd{host.docker.internal}{Resolves to you localhost IP as seen from container. In order to use that, we must add new host in build command: \\\code{---add-host=host.docker.internal:host-gateway}}
    \cmd{docker network create}{Create new docker network}
    \cmd{docker network ls}{List all networks}
  }

  \summary{Volumes \& Bind mounts}{
    \feature{Anonymous volumes}{
      \item Created specifically for a single container
      \item Survives container shutdown / restart, but \underline{NOT removal}
      \item \underline{Cannot be shared} across containers
    }
    \feature{Anonymous volumes}{
      \item Created in general / not container specific
      \item Survives container shutdown / restart \& removal
      \item \underline{Can be shared} across containers
      \item Can be reused
    }
    \feature{Bind mounts}{
      \item Located at host file system, not tied to any container
      \item Survives container shutdown / restart \& removal
      \item \underline{Can be shared} across containers
      \item Can be reused
    }
  }
\end{multicols*}

\end{document}